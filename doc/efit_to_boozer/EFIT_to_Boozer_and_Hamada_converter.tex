\documentclass[12pt]{article}
%\sloppy 
\textheight24.7cm                                                              
\textwidth18.0cm                                                               
\topmargin-1.5cm                                                               
\oddsidemargin-1.0cm                                                            
\evensidemargin0.0cm
\usepackage{times}
\usepackage{amsmath}
%\renewcommand{\baselinestretch}{2}
\newcommand{\be}[1]{\begin{equation} \label{#1}}
\newcommand{\ee}{\end{equation}}
\newcommand{\bea}[1]{\begin{eqnarray} \label{#1}}
\newcommand{\eea}{\end{eqnarray}}
\newcommand{\bean}{\begin{eqnarray*}}
\newcommand{\eean}{\end{eqnarray*}}

\newcommand{\non}{\nonumber\\}
\newcommand{\eq}[1]{(\ref{#1})}
\newcommand{\difp}[2]{\frac{\partial #1}{\partial #2}}
\newcommand{\br}{{\bf r}}
\newcommand{\bR}{{\bf R}}
\newcommand{\bA}{{\bf A}}
\newcommand{\bB}{{\bf B}}
\newcommand{\bE}{{\bf E}}
\newcommand{\bm}{{\bf m}}
\newcommand{\bn}{{\bf n}}
\newcommand{\bN}{{\bf N}}
\newcommand{\bp}{{\bf p}}
\newcommand{\bF}{{\bf F}}
\newcommand{\bz}{{\bf z}}
\newcommand{\bZ}{{\bf Z}}
\newcommand{\bV}{{\bf V}}
\newcommand{\bv}{{\bf v}}
\newcommand{\bu}{{\bf u}}
\newcommand{\bx}{{\bf x}}
\newcommand{\bX}{{\bf X}}
\newcommand{\bJ}{{\bf J}}
\newcommand{\bj}{{\bf j}}
\newcommand{\bk}{{\bf k}}
\newcommand{\bTheta}{{\bf \Theta}}
\newcommand{\btheta}{{\boldsymbol\theta}}
\newcommand{\bOmega}{{\bf \Omega}}
\newcommand{\bomega}{{\boldsymbol\omega}}
\newcommand{\brho}{{\boldsymbol\rho}}
\newcommand{\re}{{\rm e}}
\newcommand{\rd}{{\rm d}}
\newcommand{\rJ}{{\rm J}}
\newcommand{\ph}{{\varphi}}
\newcommand{\te}{\theta}
\newcommand{\tht}{\vartheta}
\newcommand{\vpar}{v_\parallel}
\newcommand{\vparkb}{v_{\parallel k b}}
\newcommand{\vparkm}{v_{\parallel k m}}
\newcommand{\Jpar}{J_\parallel}
\newcommand{\ppar}{p_\parallel}
\newcommand{\Bpstar}{B_\parallel^*}
\newcommand{\intpi}{\int\limits_{0}^{2\pi}}
\newcommand{\summ}{\sum \limits_{m=-\infty}^\infty}
\newcommand{\tb}{\tau_b(\uv)}
\newcommand{\bh}{{\bf h}}
\newcommand{\cE}{{\cal E}}
\newcommand{\cB}{{\cal B}}
\newcommand{\cg}{{\cal G}}
\newcommand{\odtwo}[2]{\frac{\rd #1}{\rd #2}}
\newcommand{\pdone}[1]{\frac{\partial}{\partial #1}}
\newcommand{\pdtwo}[2]{\frac{\partial #1}{\partial #2}}
\newcommand{\ds}{\displaystyle}
%\newcommand{\bc}{\begin{center} 
%\newcommand{\ec}{\end{center}}
%\input{prespan}

\begin{document}

\section{Generation of Boozer file from EFIT output}

\subsection{Magnetic field from EFIT data}
\label{ssec:efitdata}

Code EFIT provides axisymmetric equilibrium field which can be written as a sum of poloidal and toroidal field as follows,
\be{axifield}
\bB=\nabla\times\left(A_\varphi\nabla\varphi\right)+B_\varphi\nabla\varphi,
\ee
where the toroidal covariant vector-potential component $A_\varphi=\psi$ is the poloidal flux specified in the nodes of rectangular grid
in cylindrical coordinates, $\psi=\psi(R,Z)$, and the toroidal covariant magnetic field component $B_\varphi=B_\varphi(\psi)$ is a flux function
equal up to a factor to a poloidal current, $B_\varphi= 2 I_{\rm pol} c^{-1}$.
Interface to EFIT data, ``field\_divB0'', uses 2D splines of 5-th order to interpolate $\psi$ and provides physical cylindrical components of 
the magnetic field as functions of cylindrical variables $(R,\varphi,Z)$ according to~\eq{axifield},
\be{cylcomps}
\hat B_R = -\frac{1}{R}\difp{\psi}{Z},
\qquad
\hat B_Z = \frac{1}{R}\difp{\psi}{R},
\qquad
\hat B_\varphi = \frac{B_\varphi}{R}.
\ee
These components as well as their derivatives over all coordinates are available via formal arguments of subroutine ``field''. 
In addition, function $\psi$ and its first and second derivatives are available via module ``field\_eq\_mod''.

\subsection{Symmetry flux coordinates by field line integration}
Equations of field line in cylindrical coordinates are
\bea{flint_cyl}
\frac{\rd R}{\rd \varphi}=\frac{B^R}{B^\varphi}=\frac{R\hat B_R}{\hat B_\varphi},
\nonumber \\
\frac{\rd Z}{\rd \varphi}=\frac{B^Z}{B^\varphi}=\frac{R\hat B_Z}{\hat B_\varphi},
\eea
while the same equations in symmetry (and any other straight field line) flux coordinates are
\bea{flint_sym}
\frac{\rd r}{\rd \varphi}=\frac{B^r}{B^\varphi}=0,
\nonumber \\
\frac{\rd \vartheta}{\rd \varphi}=\frac{B^\vartheta}{B^\varphi}=\iota,
\eea
where $\iota=\iota(r)$. Integrating Eqs.~\eq{flint_cyl} over one poloidal turn what requires finding the periodic boundary 
upon integrating over $\varphi$ from 0 to this boundary, $\varphi_{\rm max}$, we obtain safety factor $q=\iota^{-1}$ as
$q=\varphi_{\rm max}(2\pi)^{-1}$.
Repeating the integration with a fixed step $\Delta\varphi=2\pi q K^{-1}$ where $K$ is the number of steps we obtain
the data on equi-distant grid of symmetry flux angle $\vartheta$ ranging in $[0, 2\pi]$, i.e. a set of array elements
$(\vartheta_k, R_k, Z_k, B_k, \sqrt{g}_k)$ belonging to the same point $\vartheta_k = 2\pi k K^{-1}$.
Metric determinant here is obtained from
\be{divB}
\nabla\cdot\bB=0 \qquad \Rightarrow\qquad \difp{}{\vartheta}\sqrt{g}B^\vartheta + \difp{}{\varphi}\sqrt{g}B^\varphi
=\difp{}{\vartheta}\sqrt{g}B^\vartheta = \iota \difp{}{\vartheta}\sqrt{g}B^\varphi=0
\ee
as
\be{sqrtg_sym}
\sqrt{g} = \frac{C_s}{B^\varphi}=\frac{C_s R}{\hat B_\varphi},
\ee
where constant $C_s=C_s(r)$ is a flux function. The value of this constant is not important since it cancels in neoclassical averages.
Note that Eq.~\eq{sqrtg_sym} is valid for general 3D field geometries.

\noindent
Repeating the integration for a set of flux surfaces $r=r_j$ we obtain the data on the rectangular grid equidistant in $\vartheta$.
This data is interpolated over $\vartheta$ with help of periodic splines and over $r$ with help of Lagrange polynomials.
Thus, continuous functions $R=R(r,\vartheta)$, $Z=Z(r,\vartheta)$, $B=B(r,\vartheta)$ and $\sqrt{g}=\sqrt{g(r,\vartheta)}$
are available everywhere in plasma volume together with their derivatives computed from splines and Lagrange polynomials.



\subsection{Boozer transformation function from field line integration}
\label{ssec:boozerdata}

For Boozer coordinates radial variable $r$ is unchanged.
Then, the Jacobian $\sqrt{g}$ of symmetry flux coordinates $(r,\vartheta,\varphi)$
is expressed through the Jacobian $\sqrt{g_B}$ of Boozer coordinates $(r,\vartheta_B,\varphi_B)$
as follows
\be{jacs}
\sqrt{g} = \difp{(\vartheta_B,\varphi_B)}{(\vartheta,\varphi)}\sqrt{g_B} \equiv J_B \sqrt{g_B}.
\ee
Substituting in~\eq{jacs} Boozer angles from direct transformation via transformation function $G=G(r,\vartheta,\varphi)$,
\be{invtrans}
\vartheta_B = \vartheta+\iota G,
\qquad
\varphi_B = \varphi+G,
\ee
magnetic differential equation for $G$ is obtained as
\be{magdifG}
\frac{\rd G}{\rd \varphi} \equiv \frac{1}{B^\varphi}\bB\cdot\nabla G \equiv \iota \difp{G}{\vartheta}+\difp{G}{\varphi}=\frac{\sqrt{g}}{\sqrt{g_B}}-1
=J_B-1.
\ee
The Jacobian of Boozer coordinates for the particular choice of radial variable $r=\psi_{\rm tor}$ is
\be{jacbooz}
\sqrt{g_B}=\frac{\iota B_\vartheta + B_\varphi}{B^2},
\ee
where $B_\vartheta=B_\vartheta(r)$ and $B_\varphi=B_\varphi(r)$ are flux functions. Therefore, for arbitrary choice of $r$ this Jacobian is
\be{jacboozarb}
\sqrt{g_B}=\frac{C_B}{B^2},
\ee
with $C_B=C_B(r)$ being a flux function too. Substituting~\eq{jacboozarb} and~\eq{sqrtg_sym} in~\eq{magdifG} this equation is transformed to
\be{magdifG_expl}
\frac{\rd G}{\rd \varphi} =\frac{R B^2}{C \hat B_\varphi}-1,
\ee
where $C=C_B C_s^{-1}$ is also a flux function. Particular expression for this function is not needed because it can be evaluated from the condition
that transformation function $G$ is a single-valued function on the flux surface,
\be{singval}
\lim_{L_\varphi \rightarrow \infty}\frac{1}{L_\varphi}\int\limits_0^{L_\varphi} \rd\varphi\left(\frac{R B^2}{C \hat B_\varphi}-1\right)
= \left\langle B^\varphi\left(\frac{R B^2}{C\hat B_\varphi}-1\right)\right\rangle \left\langle B^\varphi \right\rangle^{-1} = 0.
\ee
In axisymmetric field where $G=G(r,\vartheta)$ it is sufficient to set the upper integration limit over $\varphi$ not to infinity
but to $L_\varphi=\varphi_{\rm max}=2\pi q$ what results in
\be{Cdef}
C = \frac{1}{2\pi q} \int\limits_0^{2\pi q} \rd\varphi \frac{R B^2}{\hat B_\varphi}.
\ee
Thus, constant $C$ can be determined during the first field line integration together with safety factor $q$.
Second integration of field line equations~\eq{flint_cyl} extended by Eq.~\eq{magdifG_expl} will result in additional quantity $G_k$ 
in the set of array elements $(\vartheta_k, R_k, Z_k, B_k, \sqrt{g}_k, G_k)$ such that subsequent 2D interpolation provides $G$ continuously 
everywhere together with necessary derivatives.

\noindent
This primary information from field line integration is already sufficient for computation of mod-B spectrum, $B_{mn}$, which for axisymmetric field
contains finite amplitudes only for $n=0$,
\be{Bm0}
B_{m0}=\frac{1}{2\pi}\int\limits_0^{2\pi} \rd\vartheta_B B \re^{-i m \vartheta_B}
=\frac{1}{2\pi}\int\limits_0^{2\pi} \rd\vartheta \left(1+\iota \difp{G}{\vartheta}\right) B \re^{-i m \left(\vartheta+\iota G\right)}.
\ee
Here, one can express derivative of $G$ via~\eq{magdifG_expl} which taking into account~\eq{magdifG} and axisymmetry of $G$ results in
the explicit form of the transformation Jacobian from symmetry flux to Boozer coordinates,
\be{transfactor}
J_B=1+\iota \difp{G}{\vartheta}=\frac{R B^2}{C \hat B_\varphi}.
\ee
Integrals~\eq{Bm0} of periodic sub-integrand are efficiently evaluated on an equidistant grid of $\vartheta$,
\be{thetak}
\vartheta_k = \frac{2\pi k}{K}, \qquad k=1,2,\dots,K.
\ee
Denoting values of different function at the nodes of the grid as follows,
\be{fnodes}
J_B^{(k)} = \frac{R(r,\vartheta_k) B^2(r,\vartheta_k)}{C(r)\hat B_\varphi(r,\vartheta_k)},
\qquad
B_{(k)} = B(r,\vartheta_k),
\qquad
\cE_{(k)} = \exp\left(- i \left(\vartheta_k + \iota(r)G(r,\vartheta_k)\right)\right),
\ee
Fourier amplitudes~\eq{Bm0} are obtained as
\be{Bm0num}
B_{m0}=\lim_{K\rightarrow\infty}\frac{1}{K}\sum\limits_{k=1}^K J_B^{(k)} B_{(k)} \cE_{(k)}^m.
\ee
The same is done also for cylindrical coordinates,
\be{RZ0num}
R_{m0}=\lim_{K\rightarrow\infty}\frac{1}{K}\sum\limits_{k=1}^K J_B^{(k)} R_{(k)} \cE_{(k)}^m,
\qquad
Z_{m0}=\lim_{K\rightarrow\infty}\frac{1}{K}\sum\limits_{k=1}^K J_B^{(k)} Z_{(k)} \cE_{(k)}^m,
\ee
including transformation function $\lambda$ defined via $2\pi\lambda=\varphi-\varphi_B=-G$,
\be{lam0m}
\lambda_{m0}= - \frac{1}{2\pi}\lim_{K\rightarrow\infty}\frac{1}{K}\sum\limits_{k=1}^K J_B^{(k)} G_{(k)} \cE_{(k)}^m.
\ee
Note that NEO-2 needs these coordinates for two purposes. First, it computes effective plasma radius defined
as $\rd V= S \rd r$ where $V$ is volume limited by flux surface and $S$ is flux surface area. Second, it
uses $R$ and $Z$ for computation of radial covariant component of the magnetic field $B_r$ via metric tensor
so that this component computed by VMEC directly, which is discussed also in the next subsection, is not used.

\subsection{Covariant field components}
\label{ssec:covar}

For computation of covariant field components in Boozer coordinates 
%(actually we need only $B_r^B$ but $B_\vartheta^B$ and $B_\varphi^B$ are computed here for
%a formula check) 
we use the fact that we know cylindrical coordinates $R$ and $Z$ as functions of symmetry flux coordinates $\bx=(r,\vartheta,\varphi)$
everywhere together with their derivatives over $x^i$. Thus, we can assume that we know the covariant unit vectors 
$\partial \br / \partial x^i$ of symmetry flux coordinates since transformation from cylindrical to Cartesian coordinates is straightforward.

\noindent
Radial covariant component is obtained as
\be{B_r}
B_r^B = \bB\cdot\left(\difp{\br}{r}\right)_B = \bB\cdot\difp{(\br,\vartheta_B,\varphi_B)}{(r,\vartheta_B,\varphi_B)}
= \bB\cdot\difp{(\br,\vartheta_B,\varphi_B)}{(r,\vartheta,\varphi)}\difp{(\vartheta,\varphi)}{(\vartheta_B,\varphi_B)}
= \bB\cdot\difp{(\br,\vartheta_B,\varphi_B)}{(r,\vartheta,\varphi)}\frac{1}{J_B}.
\ee
For more explicit expressions we use axial symmetry in order to obtain
\bea{B_r_expl}
B_r^B 
&=&
\bB\cdot\difp{\br}{r}-\bB\cdot\difp{\br}{\vartheta}\frac{1}{J_B}\difp{}{r}\iota G
-\bB\cdot\difp{\br}{\varphi}\frac{1}{J_B}\left(\difp{G}{r}-G\difp{\iota}{r}\difp{G}{\vartheta}\right)
\\
&=&
\hat B_R \difp{R}{r}+\hat B_Z \difp{Z}{r}
-\left(\hat B_R \difp{R}{\vartheta}+\hat B_Z \difp{Z}{\vartheta}\right)\frac{1}{J_B}\difp{}{r}\iota G
- \hat B_\varphi \frac{R}{J_B}\left(\difp{G}{r}-G\difp{\iota}{r}\difp{G}{\vartheta}\right),
\nonumber
\eea
where $J_B$ is given by~\eq{transfactor}. Cylindrical coordinates $R$ and $Z$ and function $G$ and their derivatives 
in the last expression~\eq{B_r_expl} are known from the interpolation of field line integration result. 
Physical cylindrical components of the magnetic field in this expression should be computed by ``field'' routine
for known $R$ and $Z$. Therefore, covariant radial component~\eq{B_r_expl} is known at equidistant grid
nodes $\vartheta_k$,
\be{B_r_nodes}
B_{r(k)}^B = B_r^B(r,\vartheta_k),
\ee
and its Fourier amplitudes can be computed in analogy to~\eq{Bm0num} as
\be{Brm0num}
B_{r,m0}=\lim_{K\rightarrow\infty}\frac{1}{K}\sum\limits_{k=1}^K J_B^{(k)} B_{r(k)}^B \cE_{(k)}^m.
\ee

\noindent
Component over poloidal angle is computed in a similar way to~\eq{B_r},
\be{B_theta}
B_\vartheta^B = \bB\cdot\left(\difp{\br}{\vartheta_B}\right)_B = \bB\cdot\difp{(r,\br,\varphi_B)}{(r,\vartheta_B,\varphi_B)}
= \bB\cdot\difp{(\br,\varphi_B)}{(\vartheta,\varphi)}\frac{1}{J_B}
= \bB\cdot\left(\difp{\br}{\vartheta}-\difp{G}{\vartheta}\difp{\br}{\varphi}\right)\frac{1}{J_B}.
\ee
Eliminating poloidal derivative of $G$ with help of~\eq{transfactor} and using then explicit expression for $J_B$ we get
\be{B_theta_expl}
B_\vartheta^B = \frac{1}{J_B}\left(B_\vartheta+\frac{B_\varphi}{\iota}\left(1-J_B\right)\right)
=\frac{\iota B_\vartheta+B_\varphi}{\iota J_B} - \frac{B_\varphi}{\iota}
=\frac{B^2}{\iota J_B B^\varphi} - \frac{B_\varphi}{\iota}
=q \left(C-B_\varphi\right).
\ee
Obviously $B_\vartheta^B$ is a flux function because both, $B_\varphi$ and constant $C$ are flux functions known already after the 
first field line integration. In order to see that this is a toroidal current we rewrite~\eq{Cdef} from the form of field line integral
to the form of integral over the poloidal angle using the axial symmetry,
\be{Cdef_pol}
C = \frac{1}{2\pi} \int\limits_0^{2\pi} \rd\vartheta \frac{R B^2}{\hat B_\varphi}
= \frac{1}{2\pi} \int\limits_0^{2\pi} \rd\vartheta \frac{B^2}{B^\varphi}.
\ee
Substituting this in the last expression~\eq{B_theta_expl} we get
\bea{torcurr}
B_\vartheta^B 
=\frac{q}{2\pi} \int\limits_0^{2\pi} \rd\vartheta \left(\frac{B^2}{B^\varphi}-B_\varphi\right)
=\frac{q}{2\pi} \int\limits_0^{2\pi} \rd\vartheta \frac{B_\vartheta B^\vartheta}{B^\varphi}
=\frac{1}{2\pi} \int\limits_0^{2\pi} \rd\vartheta B_\vartheta =\frac{2}{c} I_{\rm tor},
\eea
where the last expression follows from Ampere's law and Stokes theorem.

\noindent
We do not need to repeat the ansatz for the toroidal component because in axisymmetric field both flux coordinate systems
have the same toroidal covariant vector (since $G$ is independent of $\varphi$),
\be{bothsame}
\difp{\br}{\varphi_B}=\difp{\br}{\varphi}.
\ee
Respectively, toroidal covariant field component is also unchanged,
\be{B_phi}
B_\varphi^B = B_\varphi = R \hat B_\varphi=\frac{2}{c} I_{\rm pol}.
\ee

\subsection{Computation of the cross-section area and of the toroidal flux}
\label{ssec:oneDstuff}

\noindent
Cross-section area is given by the following contour integral
\be{area1}
S_{pol}=\oint \rd Z\; R(Z) = \int\limits_0^{2\pi}\rd\vartheta \difp{Z(r,\vartheta)}{\vartheta} R(r,\vartheta)
=\int\limits_0^{2\pi q}\rd\varphi \frac{\rd Z}{\rd \varphi} R = \int\limits_0^{2\pi q}\rd\varphi \frac{R^2\hat B_Z}{\hat B_\varphi},
\ee
where last two expressions correspond to the integration along the field line (see~\eq{flint_cyl}). 
Equivalently, one can switch $R$ and $Z$ in the contour integral in order to obtain
\be{area2}
S_{pol}= - \oint \rd R\; Z(R) = - \int\limits_0^{2\pi q}\rd\varphi \frac{R Z\hat B_R}{\hat B_\varphi}.
\ee
Both formulas result in positive area in AUG cases where toroidal field is in mathematically positive direction and poloidal
field points up at the outer midplane or if both fields are reversed. In a general case, results of Eqs.~\eq{area1} and~\eq{area2}
should be multiplied with ${\rm sign}(\hat B_\varphi \hat B_Z)$ in the outer midplane.

\noindent
In absence of finite-$\beta$ effect, covariant toroidal magnetic field component $B_\varphi$ is constant in plasma volume. Respectively,
physical component $\hat B_\varphi$ is a function of $R$ only, and, therefore, toroidal flux can be computed as follows,
\be{torflux_vac}
\Psi_{\rm tor}
= - \oint \rd R \; Z(R) \hat B_\varphi\left(R\right) 
= - \int\limits_0^{2\pi q}\rd\varphi R Z \hat B_R,
\ee
which is a straightforward consequence of~\eq{area2}.

\noindent
In presence of finite-$\beta$ effect, single field line integration is not sufficient, and the flux should be integrated over flux surface label
in symmetry flux coordinates,
\be{finbeta}
\Psi_{\rm tor}(r) 
= \int\limits_{r_{\rm axis}}^r \rd r^\prime \int\limits_0^{2\pi}\rd\vartheta \difp{\br}{r}\times\difp{\br}{\vartheta}\cdot \bB
=\int\limits_{r_{\rm axis}}^r \rd r^\prime \int\limits_0^{2\pi}\rd\vartheta \sqrt{g}\; B^\varphi
= 2\pi \int\limits_{r_{\rm axis}}^r \rd r^\prime \sqrt{g}\; B^\varphi,
\ee
where we used the fact that $\sqrt{g} B^\varphi$ is a flux function. Starting field line integration from the outer midplane $Z=Z_b$ 
and using the starting major radius value $R_b$ as a flux surface label, $r=R_b$, we can express cylindrical coordinates in the vicinity 
of the starting point via flux coordinates as follows,
\be{vicinity}
R = R_b + \frac{R_b \hat B_R \vartheta}{\iota \hat B_\varphi}+O(\vartheta^2),
\qquad
Z = Z_b + \frac{R_b \hat B_Z \vartheta}{\iota \hat B_\varphi}+O(\vartheta^2),
\ee
what results for metric determinant in
\be{sqrtg}
\sqrt{g}=R\difp{(R,\varphi,Z)}{(r,\vartheta,\varphi)}=-R\difp{(R,Z)}{(r,\vartheta)}=-\frac{R_b^2 \hat B_Z}{\iota \hat B_\varphi}+O(\vartheta).
\ee
Substituting this expression evaluated exactly at the midplane, $\vartheta=0$, in~\eq{finbeta} we get
\be{finbeta_fin}
\Psi_{\rm tor}(R_b) 
= - 2\pi \int\limits_{R_{\rm axis}}^{R_b} \rd R_b^\prime\; q\left(R_b^\prime\right) R_b^\prime\;\hat B_Z(R_b^\prime,Z_b).
\ee
Since safety factor is always positive in field line integration, sign of the flux in Eq.~\eq{finbeta_fin} is correct for AUG cases 
and should be multiplied with ${\rm sign}(\hat B_\varphi \hat B_Z)$ in a general case.

\subsection{Sign conventions}
\label{ssec:signs}

\noindent
Field line integration procedure determines $q$ as a positive quantity. Therefore, direction of the poloidal angle 
corresponds to Erika's definition for positive $\sigma={\rm sign}(\hat B_\varphi \hat B_Z)$ where $\hat B_z$ is 
computed at the outer midplane.

\noindent
Poloidal current $J$ according to Erika's Fig.2 points downwards if this current is defined as a current through a 
circle centered at the main axis of the torus. This means that $J=-I_{\rm pol}$ with $I_{\rm pol}$ defined 
by~\eq{B_phi} as a current in positive direction with respect to $Z$-axis.

\noindent
Toroidal current $I$ according to Erika's Fig.2 points in mathematically positive toroidal direction.
In case $\sigma=1$ flux coordinate system $(s,\vartheta,\varphi)$ is left-handed, and, therefore, current
$I_{\rm tor}$ in Eq.~\eq{torcurr} points in negative direction with respect to the toroidal angle, and, respectively,
$I=-I_{\rm tor}$.

\noindent
If Both currents are switch the sign flux coordinate system remains left-handed, $\sigma=1$. In this case both, $B_\varphi$
and $B_\vartheta$ switch signs. If only one current switches sign, system becomes right-handed, $\sigma=-1$. Since 
toroidal angle is always counted in mathematically positive toroidal direction of cylindrical coordinates, poloidal 
angle changes direction in both such cases in order to keep $q$ positive. 
In the first such case where the poloidal current $I_{\rm pol}$ switches sign, direction of toroidal field $B_\varphi$
is switched. Since poloidal current is defined independently of coordinate system as current through the circle centered
at $Z$-axis it does not change sign but the poloidal field $B_\vartheta$ does together with the direction
of poloidal angle. In the second such case where the toroidal current $I_{\rm tor}$ switches sign, sign of the
toroidal field $B_\varphi$ is unchanged and sign of the poloidal field $B_\vartheta$ is also unchanged because
change in the direction of the poloidal field is compensated by the change of the poloidal angle direction.
Thus, in the general case of positively defined $q$ relation between currents and covariant field components can
be written as
\be{relcurfield}
I=-\sigma I_{\rm tor}, \qquad J=-I_{\rm pol},
\ee
where $I_{\rm tor}$ and $I_{\rm pol}$ are defined by Eqs.~\eq{torcurr} and~\eq{B_phi}, respectively.
Besides that, we note that toroidal flux~\eq{finbeta_fin} is in mathematically positive direction
only for the right-handed flux coordinate system, $\sigma=-1$. For the left-handed system it is counted
in mathematically negative direction. In order count it in mathematically positive direction of
cylindrical coordinates in all cases we re-define the sign of the normalized toroidal flux
$\psi_{\rm tor}=\Psi_{\rm tor}(2\pi)^{-1}$ as follows,
\be{finbeta_fin_sigma}
\psi_{\rm tor}(R_b)
= \sigma \int\limits_{R_{\rm axis}}^{R_b} \rd R_b^\prime\; q\left(R_b^\prime\right) R_b^\prime\;\hat B_Z(R_b^\prime,Z_b).
\ee

\subsection{Pressure gradient}
\label{ssec:pressure}

\noindent
For completeness of Boozer file let us compute pressure gradient. Although it is not used by NEO-2, it might be useful
for linearized ideal MHD model of RMP's. Using force balance and Ampere's law,
\be{forceamp}
c\nabla p = \bj\times\bB, \qquad 4\pi\bj=c \nabla\times\bB,
\ee
one obtains flux coordinates
\be{influx}
\frac{4\pi}{B^\varphi}\difp{p}{r}=\left(\difp{}{\varphi}+\iota\difp{}{\vartheta}\right)B_r-\difp{B_\varphi}{r}-\iota\difp{B_\vartheta}{r}.
\ee
Averaging this expression over the angles eliminates $B_r$. In Boozer coordinates and axisymmetric field result is particularly simple,
\bea{dpdr}
\difp{p}{r}
&=&
-\frac{1}{2}\left(\difp{B^B_\varphi}{r}+\iota\difp{B^B_\vartheta}{r}\right)\left(\int\limits_0^{2\pi}\frac{\rd\vartheta_B}{B^\varphi_B}\right)^{-1}
=-\frac{1}{2\left(B^B_\varphi+\iota B^B_\vartheta\right)}
\left(\difp{B^B_\varphi}{r}+\iota\difp{B^B_\vartheta}{r}\right)\left(\int\limits_0^{2\pi}\frac{\rd\vartheta_B}{B^2}\right)^{-1}
\nonumber \\
&=& - \frac{1}{4\pi\left(B^B_\varphi+\iota B^B_\vartheta\right)}
\left(\difp{B^B_\varphi}{r}+\iota\difp{B^B_\vartheta}{r}\right)\left(\frac{1}{B^2}\right)_{00}^{-1},
\eea
where $(\dots)_{00}$ means Fourier amplitude for $(m,n)=(0,0)$. Last expression is valid in a general 3D case.

\section{Hamada - Boozer converter}

Perturbation field produced by MARS code in Hamada coordinates $\left(\rho_{\rm pol},\vartheta_H,\varphi_H\right)$ 
has the following form,
\be{hamada_input}
\delta B_H\left(\rho_{\rm pol},\vartheta_H,\varphi_H\right) = \sum \limits_{m=-\infty}^\infty\sum \limits_{n=-\infty}^\infty 
B_{mn}^{(H)}\left(\rho_{\rm pol}\right)\exp\left(i\left(m\vartheta_H+n \varphi_H\right)\right),
\ee
where
\be{complex_aham}
B_{mn}^{(H)}\left(\rho_{\rm pol}\right)=B_{mn}^{(H,c)}(\rho_{\rm pol})+i B_{mn}^{(H,s)}(\rho_{\rm pol}),
\ee
$B_{mn}^{(H,c)}$ and $B_{mn}^{(H,s)}$ are real and imaginary parts of the complex Fourier amplitude
given on the equidistant grid of the normalized polidal radius $\rho_{\rm pol}=\left(\psi_{\rm pol}/\psi_{\rm pol}^a\right)^{1/2}$.
In the following we omit in the arguments the flux surface label $\rho_{\rm pol}$ which plays the role of the parameter only.

We are interested in Fourier expansion amplitudes $B_{mn}^{(B)}$ over Boozer angles $\left(\vartheta_B,\varphi_B\right)$, 
such that perturbation field has a form similar to~\eq{hamada_input},
\be{boozer_output}
\delta B_B\left(\vartheta_B,\varphi_B\right) = \sum \limits_{m=-\infty}^\infty\sum \limits_{n=-\infty}^\infty
B_{mn}^{(B)}\exp\left(i\left(m\vartheta_B+n \varphi_B\right)\right).
\ee
These amplitudes are given by direct Fourier transforms as
\be{dirtrans_H-B}
B_{mn}^{(B)} = \frac{1}{(2\pi)^2}\int\limits_{0}^{2\pi}\rd \vartheta_B \int\limits_{0}^{2\pi}\rd \varphi_B
\exp\left(-i\left(m\vartheta_B+n \varphi_B\right)\right) \delta B_B\left(\vartheta_B,\varphi_B\right).
\ee
Since perturbation is the same at the samy physical point, we have 
$\delta B_B\left(\vartheta_B,\varphi_B\right)=\delta B_H\left(\vartheta_H,\varphi_H\right)$
where $\vartheta_H=\vartheta_H\left(\vartheta_B,\varphi_B\right)$ and $\varphi_H=\varphi_H\left(\vartheta_B,\varphi_B\right)$
or vice versa. Similar to the transformation~\eq{invtrans} form the symmetry flux coordinates 
(PEST coordinates in case of axisymmetric equilibrium field) to Boozer coordinates, we can express transformation from
Hamada coordinates to Boozer coordinates as
\be{angtrans_H-B}
\vartheta_H = \vartheta_B + \iota G_{HB}\left(\vartheta_B,\varphi_B\right),
\qquad
\varphi_H = \varphi_B + G_{HB}\left(\vartheta_B,\varphi_B\right).
\ee
In the axisymmetric equilibrium field, transformation function does not depend on the toroidal angle, 
$G_{HB}\left(\vartheta_B,\varphi_B\right)=G_{HB}\left(\vartheta_B\right)$. Therefore, 
$\vartheta_H=\vartheta_H\left(\vartheta_B\right)$ does not depend on toroidal angle, 
and $\varphi_H = \varphi_B + G_{HB}\left(\vartheta_B\right)$ is a linear function of $\varphi_B$.
Thus, substituting $\delta B_H$ in the form of Fourier series~\eq{hamada_input} in~\eq{dirtrans_H-B} 
evaluation of integral over $\varphi_B$ is trivial and results on Kronecker $\delta$ (no toroidal mode coupling).
As a result, linear transformation of Fourier amplitudes couple only the poloidal modes as follows,
\bea{lintrans_H-B}
B_{mn}^{(B)} = \sum \limits_{m^\prime=-\infty}^\infty c_{mm^\prime}^{(HB)}(n) B_{m^\prime n}^{(H)},
\eea
where transformation matrix $c_{mm^\prime}^{(HB)}(n)$ is explicitly given by
\bea{chbn}
c_{mm^\prime}^{(HB)}(n)
=
\frac{1}{2\pi}\int\limits_{0}^{2\pi}\rd \vartheta_B 
\exp\left(im^\prime\vartheta_H-im\vartheta_B+in \left(\varphi_H-\varphi_B\right)\right), 
\eea
where $\varphi_H-\varphi_B=G_{HB}(\vartheta_B)$. Within the procedure of filed line integration in cylindrical
variables, it is more convenient to replace the integration over Boozer poloidal angle using direct transformation
from Hamada angles with integration in symmetry flux coordinates. For this, we express both, Boozer and Hamada angles
via symmetry flux angles $(\vartheta,\varphi)$,
\bea{booz_viasym}
\vartheta_B &=&\vartheta+\iota G_B(\vartheta), \qquad \varphi_B=\varphi+G_B(\vartheta),
\nonumber
\\
\vartheta_H &=&\vartheta+\iota G_H(\vartheta), \qquad \varphi_H=\varphi+G_H(\vartheta),
\eea
where both transformation functions correspond to the axisymmetric equilibrium. Then Eq.~\eq{chbn} takes the form
\bea{chbn_sym}
c_{mm^\prime}^{(HB)}(n)
=
\frac{1}{2\pi}\int\limits_{0}^{2\pi}\rd \vartheta\left(1+\iota \difp{G_B}{\vartheta}\right)
\exp\left(i(m^\prime-m)\vartheta+im^\prime\iota G_H-im\iota G_B+in \left(G_H-G_B\right)\right).
\eea
The inverse transformation matrix is given by a similar expression with exchange of indices $B$ and $H$.
Consistency check od so-obtained inverse transformation is presented in Appendix~\ref{sec:conscheck}.

\subsection{Coversion coefficients via field line integration}
\label{ssec:conv_via_flint}
Direc transformation function from symmetry flux coordinates to Boozer coordinates
has been obtained in Section~\ref{ssec:boozerdata} via the Jacobian~\eq{jacboozarb} of Boozer coordinates
with arbitrary flux surface label $r$. We use the same label for all flux coordinates such that the Jacobian of
Hamada coordinates takes a similar form
\be{jachamarb}
\sqrt{g_H}=C_H=C_H(r).
\ee
Then, in adition to the magnetic diferential equation~\eq{magdifG_expl} for transformation function $G_B$ to Boozer coordinates
which we re-write here in new notation as
\be{magdifG_expl_booz}
\frac{\rd G_B}{\rd \varphi} =\frac{R B^2}{C_{SB} \hat B_\varphi}-1,
\ee
transformation function to Hamada coordinates satisfies the magnetic differential equation
\be{magdifG_expl_ham}
\frac{\rd G_H}{\rd \varphi} =\frac{R}{C_{SH} \hat B_\varphi}-1.
\ee
It should be reminded that integration here is along the magnetic field line, such that all functions in~\eq{magdifG_expl_booz}
and~\eq{magdifG_expl_ham} which are independent of toroidal angle due to axial symmetry, $F=F(\vartheta)$, are the functions of
teh toroidal angle in field aligned coordinates $(\vartheta_0,\varphi)$ where 
$\vartheta_0=\vartheta-\iota\varphi$ is the field line label, $F=F(\vartheta_0 + \iota \varphi)$. To be definite, we start
field line integration at $\varphi=0$ so that $\vartheta_0$ corresponds to a starting value of usual symmetry flux angle 
$\vartheta$. Transformation functions via field line integration are then
\bea{transf_mfl}
G_B &=& G_B(\vartheta_0+\iota\varphi)=\frac{1}{C_{SB}}\int\limits_0^\varphi\rd\varphi^\prime \frac{R B^2}{\hat B_\varphi}-\varphi
+G_B^0,
\nonumber \\
G_H &=& G_H(\vartheta_0+\iota\varphi)=\frac{1}{C_{SB}}\int\limits_0^\varphi\rd\varphi^\prime \frac{R}{\hat B_\varphi}-\varphi
+G_H^0,
\eea
where $G_B^0=q(\vartheta_{B0}-\vartheta_0)$ and 
$G_H^0=q(\vartheta_{H0}-\vartheta_0)$ are integration constants setting the origin of poloidal angle in each coordinate system
so that at the same physical point $\vartheta=\vartheta_0$ respective values of poloidal angles in each coordinate system are
$\vartheta_B=\vartheta_{B0}$ and $\vartheta_H=\vartheta_{H0}$.
Constants $C_{SB}(r)$ and $C_{SH}(r)$ follow from the periodocity of transformation functions,
$G(\vartheta_0 + \iota \varphi)|_{\varphi=2\pi q}=G(\vartheta_0 + 2\pi)=G(\vartheta_0)$, as
\be{C_SBH}
C_{SB}=\frac{1}{2\pi q}\int\limits_0^{2\pi q}\rd\varphi \frac{R B^2}{\hat B_\varphi},
\qquad
C_{SH}=\frac{1}{2\pi q}\int\limits_0^{2\pi q}\rd\varphi \frac{R}{\hat B_\varphi}.
\ee

In the following, it is convenient to re-define transfomation functions via
\be{redef_via_F}
F_B(\varphi)=\int\limits_0^\varphi\rd\varphi^\prime \frac{R B^2}{\hat B_\varphi},
\qquad
F_H(\varphi)=\int\limits_0^\varphi\rd\varphi^\prime \frac{R}{\hat B_\varphi},
\ee
so that
\bea{G_vs_F}
G_B(\vartheta_0+\iota\varphi)=\frac{\varphi_{\rm max}}{F_B(\varphi_{\rm max})}F_B(\varphi)-\varphi+q(\vartheta_{B0}-\vartheta_0),
\nonumber \\
G_H(\vartheta_0+\iota\varphi)=\frac{\varphi_{\rm max}}{F_H(\varphi_{\rm max})}F_H(\varphi)-\varphi+q(\vartheta_{H0}-\vartheta_0),
\eea
and $\varphi_{\rm max} = 2\pi q$ is the toroidal field line displacement over one full poloidal turn.
With this, we can re-write
\be{F_deriv}
1+\iota \difp{G_B}{\vartheta}=1+\frac{\rd G_B}{\rd\varphi}
=\frac{\varphi_{\rm max}}{F_B(\varphi_{\rm max})} \frac{\rd F_B}{\rd\varphi},
\qquad
\frac{\rd F_B}{\rd\varphi}=\frac{R B^2}{\hat B_\varphi}.
\ee
Thus, replacing in Eq.~\eq{chbn_sym} integration over poloidal angle $\vartheta$ with integration along the field line as follows,
$$
\int\limits_{0}^{2\pi}\rd \vartheta f(\vartheta) 
= 
\iota\int\limits_{0}^{\varphi_{\rm max}}\rd \varphi f(\vartheta_0 + \iota \varphi),
$$
we substitute functions $G$ and derivative of $G_B$ via Eqs.~\eq{G_vs_F} and~\eq{F_deriv} and obtain
\bea{chbn_sym_mfl}
c_{mm^\prime}^{(HB)}(n)
&=&
\frac{1}{F_B(\varphi_{\rm max})} 
\int\limits_{0}^{\varphi_{\rm max}}\rd \varphi
\frac{\rd F_B}{\rd\varphi}
\nonumber \\
&\times&
\exp
\left(
i(m^\prime+nq)\left(\frac{2\pi F_H(\varphi)}{F_H(\varphi_{\rm max})}+\vartheta_{H0}\right)
-
i(m+nq)\left(\frac{2\pi F_B(\varphi)}{F_B(\varphi_{\rm max})}+\vartheta_{B0}\right)
\right)
\nonumber \\
&=&
\int\limits_{0}^{\varphi_{\rm max}}\rd \varphi
\frac{\rd \hat F_B}{\rd\varphi}
\exp
\left(
i(2\pi m^\prime+n\varphi_{\rm max})\hat F_H(\varphi)
-
i(2\pi m+n\varphi_{\rm max})\hat F_B(\varphi)
\right).
\eea
Easy to check that this expression results in Kronecker symbol $\delta_{mm^\prime}$ 
if we set $F_H=F_B$ and $\vartheta_{H0}=\vartheta_{B0}$.
Last expression, where
\be{hat_Fs}
\hat F_H(\varphi) = \frac{F_H(\varphi)}{F_H(\varphi_{\rm max})}+\frac{\vartheta_{H0}}{2\pi},
\qquad
\hat F_B(\varphi) = \frac{F_B(\varphi)}{F_B(\varphi_{\rm max})}+\frac{\vartheta_{B0}}{2\pi},
\ee
is evaluated by field line integration in cylindrical coordinates over the full poloidal
turn, i.e. $\varphi_{\rm max}=2\pi q$. We use discrete Fourier transform for evaluation of 
coefficients~\eq{chbn_sym_mfl} so that all we need from the field line integration are function
$\hat F_B(\varphi)$, its derivative $\rd \hat F_B(\varphi) /\rd \varphi$ and function $\hat F_H(\varphi)$
given on the equidistant grid of $\varphi$ in the interval $0 \le \varphi \le \varphi_{\rm max}$.

\subsection{Fourier series in NEO-2}
\label{ssec:neo2-spectrum}

Fourier series representation of perturbation~\eq{boozer_output} differs from representation of this field in NEO-2 where
\be{boozer_neo2}
\delta B_B\left(\vartheta_B,\varphi_B\right) = {\rm Re}\sum \limits_{m=-\infty}^\infty\sum \limits_{n=1}^\infty
B_{mn}\exp\left(i\left(m\vartheta_B+n \varphi_B\right)\right).
\ee
Let us present this series (noting that $B_{m0}^{(B)}=B_{m0}=0$) as
\bea{transeries}
\delta B_B\left(\vartheta_B,\varphi_B\right) 
&=& 
\frac{1}{2}\sum \limits_{m=-\infty}^\infty
\left(
\sum \limits_{n=1}^\infty B_{mn}\exp\left(i\left(m\vartheta_B+n \varphi_B\right)\right)
+
\sum \limits_{n=1}^\infty B_{mn}^\ast\exp\left(-i\left(m\vartheta_B+n \varphi_B\right)\right)
\right)
 \\
&=& 
\frac{1}{2}\sum \limits_{m=-\infty}^\infty
\left(
\sum \limits_{n=1}^\infty B_{mn}\exp\left(i\left(m\vartheta_B+n \varphi_B\right)\right)
+
\sum \limits_{n=-\infty}^{-1} B_{-m,-n}^\ast\exp\left(i\left(m\vartheta_B+n \varphi_B\right)\right)
\right).
\nonumber
\eea
Comparison with Eq.~\eq{boozer_output} results in relations
\bea{rel_B_neo2}
B_{mn}^{(B)} &=& \frac{1}{2}B_{mn}, \qquad n > 0,
\nonumber \\
B_{mn}^{(B)} &=& \frac{1}{2}B_{-m,-n}^\ast, \quad n < 0,
\eea
which can also be written via Heaviside fnction $\Theta(x)$ as
$$
B_{mn}^{(B)} = \frac{1}{2}\left(B_{mn}\Theta(n)+B_{-m,-n}^\ast\Theta(-n)\right).
$$
Last representation makes it easy to check that
$$
\left(B^{(B)}_{-m,-n}\right)^\ast = B_{mn}^{(B)},
$$
which should hold for Fourier coefficients of the real quantity. Thus, we need only $B_{mn}=2B_{mn}^{(B)}$ for $n>0$
for NEO-2 representation. Presenting now
\be{cossin}
B_{mn}=B_{mn}^{c}- i B_{mn}^{s},
\ee
where $B_{mn}^{c}$ and $B_{mn}^{s}$ are real coefficients,
series~\eq{boozer_neo2} takes the form
\be{boozer_neo2_cossin}
\delta B_B\left(\vartheta_B,\varphi_B\right) = \sum \limits_{m=-\infty}^\infty\sum \limits_{n=1}^\infty
\left(
B_{mn}^c\cos\left(m\vartheta_B+n \varphi_B\right)
+
B_{mn}^s\sin\left(m\vartheta_B+n \varphi_B\right)
\right),
\ee
which is used in the respective Boozer file for NEO-2.
Thus we obtained the required coefficients via $B_{mn}^{(B)}$ for $n>0$ as
\be{reqcoef}
B_{mn}^c = 2\; {\rm Re}\; B_{mn}^{(B)},
\qquad
B_{mn}^s = - 2\; {\rm Im}\; B_{mn}^{(B)}.
\ee



\appendix
\section{Consistency check of direct and inverse Boozer-Hamada conversion}
\label{sec:conscheck}

Let us compute the inverse transformation matrix $c_{m^\prime m^{\prime\prime}}^{(BH)}(n)$
using the definition~\eq{booz_viasym} with the exchange of indices $B$ and $H$
and check that the product
\be{tildel}
\tilde \delta_{mm^{\prime\prime}}
=
\sum\limits_{m^\prime=-\infty}^\infty c_{mm^\prime}^{(HB)}(n) c_{m^\prime m^{\prime\prime}}^{(BH)}(n)
\ee
is the same with Kronecker symbol. Using in the explicit expression for this product,
\bea{expl_prod}
\tilde \delta_{mm^{\prime\prime}}
&=&
\frac{1}{4\pi^2}\sum\limits_{m^\prime=-\infty}^\infty
\int\limits_{0}^{2\pi}\rd \vartheta
\int\limits_{0}^{2\pi}\rd \vartheta^\prime
\left(1+\iota \difp{G_B(\vartheta)}{\vartheta}\right)
\left(1+\iota \difp{G_H(\vartheta^\prime)}{\vartheta^\prime}\right)
\\
&\times&
\exp\left[i(m^\prime-m)\vartheta+im^\prime \iota G_H(\vartheta)
-im \iota G_B(\vartheta)+in \left(G_H(\vartheta)-G_B(\vartheta)\right)\right]
\nonumber \\
&\times&
\exp\left[i(m^{\prime\prime}-m^\prime)\vartheta^\prime
+im^{\prime\prime}\iota G_B(\vartheta^\prime)-im^\prime\iota G_H(\vartheta^\prime)+in \left(G_B(\vartheta^\prime)
-G_H(\vartheta^\prime)\right)\right]
\nonumber 
\eea
the summation formula
\be{sumformula}
\sum\limits_{m=-\infty}^\infty {\rm e}^{i m \alpha}
=2\pi\sum\limits_{k=-\infty}^\infty 
\delta\left(\alpha + 2\pi k\right)
\ee
and periodicity of the sub-integrand we extend the integration over $\vartheta^\prime$ to infinite limits as follows
$$
\sum\limits_{k=-\infty}^\infty\int\limits_{0}^{2\pi}\rd \vartheta^\prime f(\vartheta^\prime +2\pi k + g(\vartheta^\prime,\vartheta))
h(\vartheta^\prime,\vartheta)
=\int\limits_{-\infty}^{\infty}\rd \vartheta^\prime f(\vartheta^\prime+ g(\vartheta^\prime,\vartheta))
h(\vartheta^\prime,\vartheta),
$$
where $g(\vartheta^\prime,\vartheta)$ and $h(\vartheta^\prime,\vartheta)$ are periodic 
functions of $\vartheta^\prime$.
Thus, we obtain
\bea{explcheck_inv}
\tilde \delta_{mm^{\prime\prime}} &=&
\frac{1}{2\pi}
\int\limits_{0}^{2\pi}\rd \vartheta
\int\limits_{-\infty}^{\infty}\rd \vartheta^\prime
\left(1+\iota \difp{G_B(\vartheta)}{\vartheta}\right)
\left(1+\iota \difp{G_H(\vartheta^\prime)}{\vartheta^\prime}\right)
\nonumber \\
&\times&
\delta\left(\vartheta-\vartheta^\prime + \iota G_H(\vartheta)-\iota G_H(\vartheta^\prime)\right)
\nonumber \\
&\times&
\exp\left[-i m\vartheta-im\iota G_B(\vartheta)+in \left(G_H(\vartheta)-G_B(\vartheta)\right)\right]
\nonumber \\
&\times&
\exp\left[i m^{\prime\prime}\vartheta^\prime
+im^{\prime\prime} \iota G_B(\vartheta^\prime)+in \left(G_B(\vartheta^\prime)
-G_H(\vartheta^\prime)\right)\right]
\nonumber \\
&=&
\frac{1}{2\pi}
\int\limits_{0}^{2\pi}\rd \vartheta
\left(1+\iota \difp{G_B(\vartheta)}{\vartheta}\right)
\exp\left(i(m^{\prime\prime}-m)\left(\vartheta+\iota G_B(\vartheta)\right)\right)
\nonumber \\
&=&
\frac{1}{2\pi}
\int\limits_{0}^{2\pi}\rd \vartheta_B
\exp\left(i(m^{\prime\prime}-m)\vartheta_B\right)=\delta_{m m^{\prime\prime}}.
\eea



\end{document}
