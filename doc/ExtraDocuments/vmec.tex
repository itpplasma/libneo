\documentclass{article}
\usepackage{graphicx}
\usepackage{wrapfig}
\usepackage{amsmath}

\usepackage[english]{babel}
\usepackage[utf8]{inputenc}
\usepackage[T1]{fontenc}

\usepackage[a4paper]{geometry}
\usepackage{authblk}
\usepackage{setspace}
\usepackage{csquotes}
\usepackage{booktabs}
%%%%%%%%%%%%%%%%%% My preamble %%%%%%%%%%%%%%%%%%%%%%%%%%
%\usepackage[numbers, sort&compress]{natbib}
%\setcitestyle{numbers,square,comma,aysep={,},yysep={,},notesep={,}}
%\bibliographystyle{unsrtnat}
%\bibliographystyle{IEEEtranSN}

%References
\usepackage[
    bibstyle=phys,
    biblabel=brackets,
    citestyle=numeric-comp,
    isbn=false,
    doi=false,
    sorting=none,
    url=false,
    defernumbers=true,
    bibencoding=utf8,
    backend=biber,
    %maxbibnames=3,maxcitenames=3,
    %minnames=3,
    maxnames=30,
    ]{biblatex}
\setcounter{biburllcpenalty}{7000}
\setcounter{biburlucpenalty}{8000}
\addbibresource[]{eccd_ref.bib}
\DeclareBibliographyCategory{fullcited}
\newcommand{\mybibexclude}[1]{\addtocategory{fullcited}{#1}}
\DeclareFieldFormat{titlecase}{\MakeCapital{#1}}
\DeclareFieldFormat{sentencecase}{\MakeSentenceCase{#1}}

\usepackage[font=small,labelfont=bf]{caption}
\usepackage{amssymb}
\usepackage{bm}

\usepackage{../shared/Commands}

\title{\textbf{VMEC Usage}}

\begin{document}

\maketitle

\section{Introduction}
Refering to stellopt version of VMEC
Addition to the manual, answering questions not answered there.


\section{General}
Documentation at https://princetonuniversity.github.io/STELLOPT/VMEC

Not-normalized toroidal flux in variable ``phi''.
Not-normalized poloidal flux in variable ``chi''.

Pressure in variables ``presf'' (full grid) and ``pres'' (half grid).

Difference between full and half grid?
- If values are defined at grid or inbetween?

What libraries are actually needed?
The wiki lists
\begin{verbatim}
git
gfortran
openmpi-bin
libopenmpi-dev
g++
libnetcdf-dev
libnetcdff-dev
libhdf5-openmpi-dev
hdf5-tools
libblas-dev
liblapack-dev
python3
python3-numpy
python3-h5py
pgplot5
libncarg-dev
libscalapack-openmpi-dev
\end{verbatim}
(actually gfortran is listed twice in the wiki).
On faepop35 pgplot5 and libncarg-dev are missing.

Python scripts should to be changed to qt5.

With this function read\_vmec from stellopt library could be used.

Also the graphical interface could be used. Note that this has to be
started in a folder with a vmec output file, with every file that starts
with ``wout'' is considered. Files which folow this naming convention but
are not vmec files might cause problems.
There seem to be some bugs in the graphical interface. In the ``cut''
plots y-axis seems to be incorrect (label says it is normalized, with
values $> 1$).

Conversion from vmec format to boozer file with script boozer.py. This
requires name of vmec output file and the number of flux surfaces.


\section{compiling}

\subsection{on marconi}
Load the required modules for the intel compiler (see online
documentation, list of required modules is on the the marconi subpage).

Set the environment variable $STELLOPT_PATH$ to point to the location to
which you cloned stellopt.

Loading modules and setting $STELLOPT_PATH$ is best done in $.bashrc$.
If you do this, remember that either a new login is required or you have
to source ~/.bashrc, otherwise the changes will not take effect.


\section{Convert files from vmec to boozer}
Conversion of files from vmec to boozer format is done in two steps.
First step is the actual conversion, second is extraction of toroidal
modes into seperate files.

First step is done with the python function convert_to_boozer
from the file boozer.py, which will convert a single flux surface.
This file can also be called as script, with the name of the input file
and the the number of flux surfaces minus one as arguments. Example:
\begin{verbatim}
boozer.py vmec_input.nc 49
\end{verbatim}
This will result in a single file with all the poloidal and toroidal
modes. \neotwo requires separation of toroidal modes, in axisymmetric
background ($n = 0$) and perturbation ($n > 0$).
Creating individual files  with toroidal perturbations can be done with
the octave function
\begin{verbatim}
extract_pert_field(file_base, file_ext, file_descr)
\end{verbatim}

Default value for $file\_descr$ does not work. (Why?) Thus ' ' needs to
be passed.


\section{Makegrid}
The program ``Makegrid'', part of STELLOPT, allows to create coil-field
files, required by vmec.
Makegrid itself requires a file with the coil geometry to work (see
stellopt documentation for file format, coordinates in the input file
are xyz(?)). The code is usually
interactive, it asks some questions about the configuration it should
use. However, the answers can be written to a file (one line for each),
in the respective order, and the file then provided to makegrid via
standard input.
Note that the number of toroidal cuts selected must be a multiple of the
nzeta parameter in the vmec input file.
The region determined here by minimum/maximum of radius and vertical
position, must match the extend of the plasma in vmec. The extend of the
plasma in vmec might not have anything to do with the real plasma
extend. It is better to make the grid larger than needed, because if it
is to small, vmec will complain, but continue, so it is unclear if this
actually affects the results (but I would think so).

For RMP simulations, with 3D fields the ``periods'' parameter in the
coil file should be one, as each RMP coil needs to be in a different
coil group.


\section{FAQ}
\paragraph I get ``ARNORM OR AZNORM EQUAL ZERO IN BCOVAR'' after message
``TRYING TO IMPROVE INITIAL MAGNETIC AXIS GUESS''.

You can try to decrease the number of radial points for the first
stages, this seems to have helped in a DEMO case.

\paragraph Minor and major radius are not set in the output file.

?

\end{document}
