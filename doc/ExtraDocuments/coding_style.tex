\documentclass{article}
\usepackage{graphicx}
\usepackage{wrapfig}
\usepackage{amsmath}

\usepackage[english]{babel}
\usepackage[utf8]{inputenc}
\usepackage[T1]{fontenc}

\usepackage[a4paper]{geometry}
\usepackage{authblk}
\usepackage{setspace}
\usepackage{csquotes}
\usepackage{booktabs}
%%%%%%%%%%%%%%%%%% My preamble %%%%%%%%%%%%%%%%%%%%%%%%%%
%\usepackage[numbers, sort&compress]{natbib}
%\setcitestyle{numbers,square,comma,aysep={,},yysep={,},notesep={,}}
%\bibliographystyle{unsrtnat}
%\bibliographystyle{IEEEtranSN}

%References
\usepackage[
    bibstyle=phys,
    biblabel=brackets,
    citestyle=numeric-comp,
    isbn=false,
    doi=false,
    sorting=none,
    url=false,
    defernumbers=true,
    bibencoding=utf8,
    backend=biber,
    %maxbibnames=3,maxcitenames=3,
    %minnames=3,
    maxnames=30,
    ]{biblatex}
\setcounter{biburllcpenalty}{7000}
\setcounter{biburlucpenalty}{8000}
\addbibresource[]{eccd_ref.bib}
\DeclareBibliographyCategory{fullcited}
\newcommand{\mybibexclude}[1]{\addtocategory{fullcited}{#1}}
\DeclareFieldFormat{titlecase}{\MakeCapital{#1}}
\DeclareFieldFormat{sentencecase}{\MakeSentenceCase{#1}}

\usepackage[font=small,labelfont=bf]{caption}
\usepackage{amssymb}
\usepackage{bm}

\usepackage{../shared/Commands}

\title{\textbf{AG Plasma Coding Style\\August 2020}}

\begin{document}

\maketitle

%%%%%%%%%%%%%%%%%%%%%%%%%%%%%%%%%%%%%%%%%%%%%%%%%%%%%%%%%%%%%%%%%%%%%%%%
%%%%%%%%%%%%%%%%%%%%%%%%%%%%%%%%%%%%%%%%%%%%%%%%%%%%%%%%%%%%%%%%%%%%%%%%
\section{Purpose of this document}
The purpose of this document is to provide a style guide for writing new
code and editing existing code of NEO-2 and related scripts and
programs.

It is clear that the code does not follow this standard everywhere (or
maybe not even anywhere), but to say this again, this is intended for
future code and rewriting/editing of existing code.

Usually things are said only once, so check all relevant sections.
Note also that some things are only relevant to a certain tool/language
(e.g. the git section) while others should be followed in general (e.g.
the section about whitespace).

%%%%%%%%%%%%%%%%%%%%%%%%%%%%%%%%%%%%%%%%%%%%%%%%%%%%%%%%%%%%%%%%%%%%%%%%
%%%%%%%%%%%%%%%%%%%%%%%%%%%%%%%%%%%%%%%%%%%%%%%%%%%%%%%%%%%%%%%%%%%%%%%%
\section{general}
Language (for names, documentation etc.) is british english.

%%%%%%%%%%%%%%%%%%%%%%%%%%%%%%%%%%%%%%%%%%%%%%%%%%%%%%%%%%%%%%%%%%%%%%%%
%%%%%%%%%%%%%%%%%%%%%%%%%%%%%%%%%%%%%%%%%%%%%%%%%%%%%%%%%%%%%%%%%%%%%%%%
\section{units}
\neotwo uses cgs. Input might be expected in SI units (due
to using external format, e.g. boozer files for magnetic field).

RB: I support change to SI units. As input/output and internal for new
codes.

%%%%%%%%%%%%%%%%%%%%%%%%%%%%%%%%%%%%%%%%%%%%%%%%%%%%%%%%%%%%%%%%%%%%%%%%
%%%%%%%%%%%%%%%%%%%%%%%%%%%%%%%%%%%%%%%%%%%%%%%%%%%%%%%%%%%%%%%%%%%%%%%%
\section{\libneo vs. specific codes}
Codes which might be of general use should be put into \libneo.

If code appears already in different programs this can be seen as a
clear sign that it should be put into \libneo.

Another big hint is, if the piece of code is not specific to plasma
physics.

If it is not clear, ask people which use/develop other codes in the
group, if they think this might be of use.
If not make it specific, and move the code later if necessary.

%%%%%%%%%%%%%%%%%%%%%%%%%%%%%%%%%%%%%%%%%%%%%%%%%%%%%%%%%%%%%%%%%%%%%%%%
%%%%%%%%%%%%%%%%%%%%%%%%%%%%%%%%%%%%%%%%%%%%%%%%%%%%%%%%%%%%%%%%%%%%%%%%
\section{code}

Use newest widely supporty standard (Fortran: 2008; c: 2009?; c++: 2011?), in
the case of fortran with files in free form.

Code should compile without warnings.

For classes ``protected'' should be the default acess for methods
(subroutines/functions of the class) and ``private'' for members
(variables).

%%%%%%%%%%%%%%%%%%%%%%%%%%%%%%%%%%%%%%%%%%%%%%%%%%%%%%%%%%%%%%%%%%%%%%%%
\subsection{naming}
Use speaking, descriptive names, make them as long as needed but not
longer than needed (and not longer than allowed by the standard in the
case of fortran).

Avoid (re-)using names that are already in use for non-local (used in
more than one file) variables/constants.

%%%%%%%%%%%%%%%%%%%%%%%%%%%%%%%%%%%%%%%%%%%%%%%%%%%%%%%%%%%%%%%%%%%%%%%%
\subsection{comments}
(Note: this subsection assumes, doxygen is used for automatic generation
of documentation. If we decide on something different, then this should
be adapted.)

Fortran: For documentation of a variable on the same line use '!<'.
Documentation for subroutines/classes/modules etc., should be put in
front of the subroutine/class/module etc., and be started with '!>' and
continued on following lines with '!>'.

Comments should not repeat what the code does (exception see next
point), but explain why the code was written as it is.

An exception of the no-repetition clause holds for equations, as these
might differ between what they would look like in print versus what they
look like in code, but of course the comment should not only repeat the
equation but make a connection between the ``symbols'' in the code and
those in print.

Exceptions/derivations made from the coding style, e.g. naming
conventions, should usually be explained with a comment.

%%%%%%%%%%%%%%%%%%%%%%%%%%%%%%%%%%%%%%%%%%%%%%%%%%%%%%%%%%%%%%%%%%%%%%%%
\subsection{Whitespace}
No tabs, except for makefiles, where it is required for rules.

Intendation is two whitespaces for every level.

contains on same level as module/subroutine/function statement
new level due to
\begin{itemize}
  \item module
  \item function/subroutine
  \item if/elseif/else
  \item do/do while
  \item case (each case is on same level as select)
\end{itemize}

Whitespace after if and before then, i.e. ``if () then''.

(Fortran) whitespace after end, e.g. ``end do'', ``end if''.

No trailing whitespace. This also means no lines with only whitespace.

%%%%%%%%%%%%%%%%%%%%%%%%%%%%%%%%%%%%%%%%%%%%%%%%%%%%%%%%%%%%%%%%%%%%%%%%
\subsection{Line length}
Lines should be limited to a maximum of 72 characters, including the
whitespace at the beginning of the line.

%%%%%%%%%%%%%%%%%%%%%%%%%%%%%%%%%%%%%%%%%%%%%%%%%%%%%%%%%%%%%%%%%%%%%%%%
\subsection{Fortran}
Use 'F90' as extension for fortran files, that require preprocessing and
'f90' for files which do not require preprocessing.

In all modules/functions/subroutines use ``implicit none''.

Use only what is needed, i.e. use ``use module, only :''.

Use ``private'' as default acess for modules, i.e. only what is
explicitly declared as public can be acessed from outside the module.

Put new files in the COMMON directory, unless the new code is definitly
specific to one of the two programs.
If you write something for one code and it might be also used in the
other, then put it into COMMON. Make sure that your code is general
enough to easily use it also from the other program, even if you do not
implement the code for this.

While fortran is not case sensitive, write code as if it is.

Use lowercase for commands.

Put the continuation sign at the end of the line and as first
non-whitespace character of the next line.

Array indices should follow the fortran convention and start at 1,
unless there are good reason do deviate (e.g. to include boundary
points, where 0 would be the first point outside the boundary, -1 the
second, etc.).

%%%%%%%%%%%%%%%%%%%%%%%%%%%%%%%%%%%%%%%%%%%%%%%%%%%%%%%%%%%%%%%%%%%%%%%%
%%%%%%%%%%%%%%%%%%%%%%%%%%%%%%%%%%%%%%%%%%%%%%%%%%%%%%%%%%%%%%%%%%%%%%%%
\section{git}

Do no rewrite published (pushed) history.

Make use of branches for general development.

Use feature branches for developments that will take more than two or
three commits over the course of a day or two.

The main branch should always compile.

The main branch should only get direct commits from bugfixes and from
merges of develop.

``develop'' shoud be branch for development, i.e.feature branches should
branch from develop and be merged back into develop.

Merging develop into master should be done regularly/on specific
occasions, e.g. submitting a paper with that code version?

Make small but complete commits, containing only one change, i.e. adding
a subroutine (maybe with changes for calling it) or fixing one bug.

After each commit to the main develop branch testcases should pass.

After a push testcases must pass.

set user.name and user.email correctly.

example
\begin{verbatim}
[user]
  name = Max Musterman
  email = musterman@tugraz.at
\end{verbatim}

Suggested (required?) to set
\begin{verbatim}
set push.default = simple
set pull.rebase = preserve
\end{verbatim}

Make a pull before you push.

It is suggested to activate the standard pre-commit hook. To do so
simply rename the file ``.git/hooks/pre-commit.sample'' to
``./git/hooks/pre-commit''.

%%%%%%%%%%%%%%%%%%%%%%%%%%%%%%%%%%%%%%%%%%%%%%%%%%%%%%%%%%%%%%%%%%%%%%%%
\subsection{tag}

Use tags for commits where usefull.

For main: version number?

%%%%%%%%%%%%%%%%%%%%%%%%%%%%%%%%%%%%%%%%%%%%%%%%%%%%%%%%%%%%%%%%%%%%%%%%
%%%%%%%%%%%%%%%%%%%%%%%%%%%%%%%%%%%%%%%%%%%%%%%%%%%%%%%%%%%%%%%%%%%%%%%%
\section{Tests}
Code should be tested.

This includes benchmarks against other codes, but also test of parts
(e.g. subroutines/functions) of the code.

The latter is intended to verify that the code still runs correctly,
after it is written and after changes (e.g. refactoring) have been made.

%%%%%%%%%%%%%%%%%%%%%%%%%%%%%%%%%%%%%%%%%%%%%%%%%%%%%%%%%%%%%%%%%%%%%%%%
%%%%%%%%%%%%%%%%%%%%%%%%%%%%%%%%%%%%%%%%%%%%%%%%%%%%%%%%%%%%%%%%%%%%%%%%
\section{(Python or Shell) Scripts}
Scripts that are either requied by neo2 itself, the make-system or the
test system should be put in the corresponding folder of neo2 and added
to the repositiory.

Treat them as code in terms of coding style.

Make them executable, and in the case of python scripts, allow to run
them as main program.

See that the code is reusable, i.e. if a function is required in
multiple scripts, see to put it into an own file which is
included/imported.

\end{document}
